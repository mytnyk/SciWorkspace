
\documentclass[12pt]{report}

\usepackage[centertags]{amsmath}
\usepackage{amsfonts, amssymb, amsthm}

%\usepackage{psfig}
\usepackage{capt-of}
%\usepackage[cp1251]{inputenc}
%\usepackage[T2A]{fontenc}
%\usepackage[ukrainian]{babel}
\usepackage{polyglossia}
\setmainlanguage{ukrainian}
\usepackage{fontspec}
\setmainfont{Times New Roman}

\begin{document}

$$
f\left(x\right)=\mathcal{GP}\left(m\left(x\right),k\left(x,x^\prime\right)\right)
$$

$$
m\left(\mathbf{x}\right)=\mathrm{E}\left[f\left(\mathbf{x}\right)\right]
$$

$$
k\left(\mathbf{x},\mathbf{x}^\prime\right)=\mathrm{E}\left[\left(f\left(\mathbf{x}\right)-m\left(\mathbf{x}\right)\right)\left(f\left(\mathbf{x}^\prime\right)-m\left(\mathbf{x}^\prime\right)\right)\right]
$$

Під час економічної кризи в Греції та на Кіпрі біткоїн перетворився на засіб захисту активів. 
Деякі власники великих капіталів, побоюючись заморожування або конфіскації своїх коштів, 
використовували біткоїн для виведення капіталу за межі національних банківських систем.

У березні 2013 року, після того як кіпрські банки виявилися на межі банкрутства, 
країна звернулася до Європейського Союзу та Міжнародного валютного фонду за фінансовою допомогою. 
ЄС, ЄЦБ і МВФ (так звана «Трійка») погодилися надати Кіпру фінансову допомогу, але за умовами, 
які включали впровадження механізму bail-in. Серед умов bail-in наприклад була так звана ``стрижка'' депозитів, 
внаслідок якої вкладники з депозитами понад 100 000 євро втратили значну частину своїх коштів. 

Попит на біткоїн зріс через його використання як засобу уникнення втрат і контролю з боку банківських установ і урядів. 
У березні 2013 року, в період найбільшої паніки серед кіпрських вкладників, 
ціна біткоїна різко зросла з приблизно 50 до понад 200 доларів США. 
Це було прямим наслідком того, що біткоїн став розглядатися не просто як технологічна цікавість 
чи засіб для проведення транзакцій в інтернеті, але як реальна альтернатива національним валютам в умовах нестабільності.


Під час фінансової кризи на Кіпрі в 2013 році біткоїн зазнав значного зростання популярності та вартості. Це було пов'язано з тим, що багато вкладників втратили довіру до традиційної банківської системи через впровадження механізму bail-in, який передбачав списання частини депозитів для порятунку банків. У результаті, люди шукали альтернативні способи зберігання та переведення коштів, і біткоїн став однією з таких альтернатив.

Згідно з даними, під час кризи на Кіпрі ціна біткоїна зросла з приблизно $30 до понад $200 за монету. Це зростання було спричинене підвищеним попитом на криптовалюту як засіб захисту капіталу від ризиків, пов'язаних з банківською системою.

Детальніше про вплив кіпрської кризи на ріст біткоїна можна прочитати в статті на BBC News Україна
[https://www.bbc.com/ukrainian/features-57526593]

Реальність така, що ціна біткоїну може як зростати вибуховими темпами у ціні, так і стрімко падати. У квітні ця криптовалюта досягла рекордного рівня, коли ціна злетіла до 64 870 доларів.

Але протягом місяця біткоїн втратив половину своєї вартості внаслідок жорсткого обвалу, викликаного двома серйозними ударами.
Першим було повідомлення від Ілона Маска, генерального директора Tesla. В середині травня він оголосив, що не прийматиме біткоїни як засіб оплати за свої автомобілі через забруднення довкілля, яке, на його думку, викликає майнінг криптовалют.

Другий удар біткоїну завдали кількома днями пізніше, коли уряд Китаю запровадив нові правила для транзакцій з криптовалютами.

Ціна впала приблизно до 30 000 доларів США, хоча в наступні дні вона частково повернула втрачені позиції (станом на 18 червня ціна 1 біткоїну становила приблизно 37 800 доларів. - Ред.)

Деякі експерти кажуть, що ці цикли підйомів та спадів триватимуть ще довго. І тому ті, хто фактично інвестує у віру в те, що вони можуть швидко розбагатіти, можуть багато втратити.




Історія з акціями GameStop на початку 2021 року стала однією з найгучніших фінансових подій останніх років. Це явище показало силу нових онлайн-спільнот, таких як форум Reddit, у взаємодії з фінансовими ринками. Акції компанії, яка переживала фінансові труднощі, раптово злетіли до небачених висот через скоординовані дії роздрібних інвесторів, що протистояли великим хедж-фондам. 
GameStop — американська роздрібна компанія, що спеціалізується на продажу відеоігор, ігрових приставок та аксесуарів. У минулому компанія була лідером на ринку продажу відеоігор, але з часом, через розвиток цифрової дистрибуції та онлайн-платформ (таких як Steam та PlayStation Store), її популярність і прибутки почали різко зменшуватися. GameStop зіткнулася з труднощами: закриття магазинів, зниження продажів, падіння ринкової вартості та збитки.

Багато великих інституційних інвесторів і хедж-фондів почали робити ставку на те, що акції GameStop продовжуватимуть падати. Вони масово використовували стратегію коротких продажів (short selling), продаючи акції компанії, яких вони не мали, у надії купити їх пізніше за нижчою ціною і отримати прибуток. У певний момент більше ніж 100\% вільно торгованих акцій GameStop були взяті в позицію на короткі продажі, що зробило компанію надзвичайно вразливою до «short squeeze» — раптового зростання ціни через масове закриття коротких позицій.

Користувачі r/WallStreetBets організували скоординовані покупки акцій GameStop, що призвело до їх стрімкого зростання. Спільнота сприймала це як можливість не тільки отримати прибуток, але й «покарати» хедж-фонди, які масово ставили на падіння акцій компанії, та створити сильний «short squeeze».

Коли ціна акцій GameStop почала зростати, хедж-фонди, які були у коротких позиціях, зазнали величезних збитків. Вони змушені були викуповувати акції, щоб закрити свої позиції, що додатково штовхало ціну ще вище. Всього за кілька днів у січні 2021 року ціна акцій GameStop піднялася з близько \$20 до понад \$350 за акцію.



----------------------------
Перше різке зростання популярності та вартості біткоїну відбулося під час фінансової кризи на Кіпрі в 2013 році. Одним із ключових факторів, які спровокували кризу на Кіпрі, була криза в Греції. Кіпрські банки мали великі інвестиції в грецькі державні облігації, які втратили свою вартість через реструктуризацію грецького боргу в рамках угоди з ЄС у 2012 році. Це призвело до значних збитків для кіпрських банків. У березні 2013 року, після того як кіпрські банки виявилися на межі банкрутства, країна звернулася до Європейського Союзу та Міжнародного валютного фонду за фінансовою допомогою. ЄС, ЄЦБ і МВФ погодилися надати Кіпру фінансову допомогу, але за умовами, які включали впровадження механізму bail-in. Серед умов bail-in наприклад була так звана «стрижка» депозитів, внаслідок якої вкладники з депозитами понад 100 000 євро втратили значну частину своїх коштів. Після оголошення про процедуру bail-in, уряд Кіпру ввів жорсткі обмеження на зняття коштів та міжнародні перекази. Це означало, що вкладники не могли просто перевести свої гроші на інші рахунки за кордон або зняти великі суми готівкою. Біткоїн, як децентралізована криптовалюта, дозволяв обійти ці обмеження, оскільки його можна було купити і зберігати без участі банківської системи. В період найбільшої паніки серед кіпрських вкладників, ціна біткоїна різко зросла з приблизно 30 до понад 200 доларів США. Це було прямим наслідком того, що біткоїн став розглядатися не просто як технологічна цікавість чи засіб для проведення транзакцій в інтернеті, але як реальна альтернатива національним валютам в умовах нестабільності. Детальніше про вплив кіпрської кризи на ріст біткоїна можна прочитати в статті на BBC News Україна [1].

\end{document}