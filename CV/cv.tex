%
% Inspired by template found at latextemplates.com by Alessandro Plasmati 
% and modified by Nicola Fontana
%
\documentclass{tccv}

\usepackage{polyglossia}
\setmainlanguage{ukrainian}
\usepackage{fontspec}
\setmainfont{Times New Roman}


\hyphenation{пас-пор-ту}

\begin{document}

\part
[Науковець, дослідник, досвічений розробник програмного забезпечення.]
{Митник Олег Юрійович}

\section{Досвід роботи}

\begin{eventlist}

\item{2023 -- 2024}
     {Materialise NV}
     {Архітектор Програмних Систем}

Дизайн та розробка бек-енд сервісу для ліцензування хмарних продуктів на основі Flexera.

\item{2019 -- 2023}
     {Materialise NV}
     {Інженер розробник}

Розробка бек-енд мікросервісів для Materialise Machine Manager.

\item{2010 -- 2019}
     {Materialise NV}
     {Лектор}

Основний лектор і тренер в літніх тренувальних програмах Materialise Academy в 2010, 2013, 2015, 2019 роках з розробки десктопних програм на C++ та C\#.

\item{2016 -- 2019}
     {Materialise NV}
     {Архітектор}

Дизайн та розробка програмного забезпечення Build Processor систем для підготовки 3D моделей до друку на 3D принтерах різних технологій.

\item{2004 -- 2016}
     {Materialise NV}
     {Інженер дослідник}

Розробка програмного забезпечення для проєкту RSM (Rapid Shell Modelling) компанії Phonak AG.
3D моделювання слухового апарату.

\item{2001 -- 2004}
     {Cyan Soft Ltd}
     {Інженер розробник}

Розробка програмного забезпечення онлайн proofing системи для підготовки і перевірки PDF документів перед публікацією.

\item{2000 -- 2001}
     {ВАТ КП ОТІ. ЄДАПС}
     {Інженер розробник}

Розробка програмного забезпечення цифрового паспорту для Єдиної Державної Автоматизованої Паспортної Системи.

\end{eventlist}

\personal
    [www.linkedin.com/in/oleg-mytnyk-1983b019/]
    {Kyiv, Ukraine}
    {+380 50 687 3202}
    {oleg.mytnyk@gmail.com}

\section{Освіта}

\begin{yearlist}

\item[Кандидат технічних наук]{2002 -- 2008}
     {Інформаційні технології}
     {НТУУ КПІ, Київ}

\item[Магістр]{1996 -- 2002}
     {Інтелектуальні системи прийняття рішень}
     {НТУУ КПІ, Інститут Прикладного Системного Аналізу, Київ}


\end{yearlist}


\section{Основні публікації}

\begin{yearlist}

\item{2008}
     {Інформаційні технології синтезу робастних нейронечітких моделей стохастичних процесів}     
     {дис... канд. техн. наук: 05.13.06}
\item{2008}
     {Balanced Neurofuzzy Models}
     {2nd International Conference on Inductive Modelling}

\item{2007}
     {Construction of Bayesian support vector regression in the feature space spanned by Bezier-Bernstein polynomial functions}
     {Cybernetics and Systems Analysis, 2007, Volume 43, Number 4, Page 613}

\end{yearlist}

%\vspace{-6pt} % Otherwise it just falls onto the next page.
\section{Навички}

\begin{factlist}

\item{Англійська}{вільно}
\item{Програмування}
     {C, C++, C\#, Python, Latex}

\item{Інструменти}
     {Matlab, colab.research, Docker, AWS}

\item{Наукові інтереси}
     {машинне навчання, стохастичні процеси, гаусівські процеси, байєсівський аналіз, нейронечіткі моделі, штучний інтелект}
\end{factlist}

\end{document}
